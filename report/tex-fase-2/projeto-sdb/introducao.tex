\chapter{Introdução} \label{chap:intro} %referência cruzada
Na unidade curricular de \textbf{Systems Deployment \& Benchmarking} foi-nos proposto um trabalho prático de análise, deployment e benchmarking de uma aplicação.Ao longo deste relatório apresentam-se os processos efetuados para a concretização do projeto. \newline 

Inicialmente faz-se uma contextualização do problema e apresentação dos objetivos, descreve-se a arquitetura e componentes da aplicação, identifica-se e apresenta-se as ferramentas utilizadas na instalação e deployment da mesma. \newline 

Por fim, e não menos importante, identificam-se as ferramentas utilizadas na monitorização e benchmark, bem como se faz uma análise dos resultados obtidos.

\chapter{Contextualização do Problema}

\section{Objetivos}
Os objetivos centrais deste projeto são analisar a arquitetura de uma aplicação, para se perceber os possíveis pontos críticos, e que soluções existem, para reduzir o impacto dos mesmos. Com a resolução destes problemas, consegue-se uma maior disponibilidade da aplicação. Do mesmo modo, monitorizar e analisar o desempenho da aplicação, para diferentes configurações dos componentes, para se entender o impacto que estas tem no desempenho da aplicação.

\section{Aplicação}

\begin{figure}[h!]
\centering
\includegraphics[scale=0.30]{images/zulip.png}
\caption{Logotipo do \href{https://zulipchat.com}{Zulip}.}
\label{img:docker}
\end{figure}
\newline

Assim, a aplicação escolhida pela equipa foi o \textbf{\href{https://zulipchat.com}{Zulip}}, na medida em que segue os requisitos do trabalho prático, bem como, tem uma boa documentação.

O \textbf{Zulip} é uma aplicação de chat em tempo real, onde o principal objetivo é oferecer um boa experiência a organizações, empresas e projetos voluntários, de pequenas equipas de amigos, até dezenas de milhares de utilizadores.\\
